% By Ccucumber12
\providecommand{\tightlist}{\setlength{\itemsep}{0pt}\setlength{\parskip}{0pt}}
\setcounter{secnumdepth}{0}

% Time Limit: 1 s \\
% Memory Limit: 1048576 KB
% \vspace{-15pt}

\subsection{Problem Description}\label{problem-description}

\vspace{6pt}

Do you recall the game developer, Little Cucumber, who recently launched his game? He invited a magical fairy to give it a try, but to his surprise, the fairy cast a bizarre rotating spell on the game... (\href{https://www.youtube.com/watch?v=dQw4w9WgXcQ}{See more})

Long story short, you have to maintain a \textcolor{red}{multiset} of strings $S$. 
At the beginning, $S$ has $N$ strings. Let $t_i$ denote the $i$-th string in $S$, $1 \leq i \leq N$. 
Then, $Q$ operations will be performed on $S$, where each operation can be one of the following two:
%Initially, there are $N$ strings in $S$, the $i$-th string is $t_i$. There are $Q$ operations, each operation is one of the following two:
\begin{enumerate}
    \item \verb|insert(|$t_j$\verb|)|: insert the string $t_j$ into $S$.
    \item \verb|remove(|$t_j$\verb|)|: remove \textcolor{red}{one instance of} the string $t_j$ from $S$, it is guaranteed that \textcolor{red}{at least one} $t_j$ is in $S$. 
\end{enumerate}

Consider two strings, $A$ and $B$, of the same length $L$.
$A$ and $B$ are \textit{rotationally identical} if and only if 
%Two strings, $A$ and $B$, are rotationally identical if and only if 
there exists $k, 1 \leq k \leq L$ such that
$$ 
A_1A_2\dots A_L = B_k B_{k+1}\dots B_L B_1\dots B_{k-1} \quad \footnote{$k=1$ is a special case representing no rotation of $B$.}
$$
, i.e., if $B$ is rotated by $k$ characters, then $A$ and $B$ are identical.

%for some $k$ (i.e. $A$ and $B$ are identical after $B$ is rotated by some characters.)

You are asked to output \underline{\textit{the number of string pairs}} in $S$ 
which are \textit{rotationally identical} before the first operation and after each operation. 



\subsection{Input}\label{input}

The first line contains two integers $N$ and $Q$, the number of strings inside $S$ initially and the number of operations, respectively. Then, the next $N$ lines have $t_i$ in order, $1 \leq i \leq N$, where $t_i$ is a string in $S$ initially. Each of the next $Q$ lines contains an integer $P$ and a string $t_j$. 

\begin{itemize}
    \item If $P = 1$, it represents an \verb|insert(|$t_j$\verb|)| operation.
    \item If $P = 2$, it represents a \verb|remove(|$t_j$\verb|)| operation.
\end{itemize}

\subsection{Output}\label{output}

The output should consist of $Q+1$ lines. The first line should contain an integer representing the number of rotationally identical string pairs in $S$ initially. Each of the next $Q$ lines should also contain an integer representing the number of rotationally identical string pairs in $S$ after each operation. 

\subsection{Constraints} \label{constraint}
\begin{itemize}
\tightlist
    \item $2 \le N \le 10^6$
    \item $0 \le Q \le 10^6$
    \item $1 \le |t_i| = |t_j| = M \le 10^6$
    \item $t_i$ and $t_j$ consists of only lowercase Latin letters.
    \item $(N+Q)M \le 10^6$ (the total string length does not exceed $10^6$.)
\end{itemize}

\subsubsection{Subtask 1 (10 pts)}\label{subtask-1}

\begin{itemize}
\tightlist
\item $N, Q, M \le 50$
\end{itemize}

\subsubsection{Subtask 2 20 pts)}\label{subtask-2}

\begin{itemize}
\tightlist
\item $Q = 0$ 
\end{itemize}

\subsubsection{Subtask 3 (20 pts)}\label{subtask-3}

\begin{itemize}
\tightlist
\item $M = 3$
\end{itemize}

\subsubsection{Subtask 4 (50 pts)}\label{subtask-3}

\begin{itemize}
\tightlist
\item No other constraints.
\end{itemize}

\subsection{Hints}

%The problem setter and testers had come up with three different solutions. You are encourage to try whatever you think may be possible, and keep on trying different approaches after receiving an \textcolor{green}{AC}. GLHF :-)
Consider the algorithm you developed for problem 1.2, which calculates the Rabin-Karp hashes for all rotations of a string. Think about how to compare between the hashes of all rotations of two strings.

\newpage

\subsection{Sample Testcases}

\subsubsection{Sample Input 1}\label{sample-input-1}
\begin{verbatim}
6 0
ntu
tun
unt
sad
dsa
qqq
\end{verbatim}

\subsubsection{Sample Output 1}\label{sample-output-1}
\begin{verbatim}
4
\end{verbatim}

\subsubsection{Sample Input 2}\label{sample-input-2}
\begin{verbatim}
3 7
bananana
nananaba
cucumber
1 nanabana
1 nabanana
1 anananab
2 cucumber
1 cumbercu
1 bercucum
2 nanabana
\end{verbatim}

\subsubsection{Sample Output 2}\label{sample-output-2}
\begin{verbatim}
1
3
6
10
10
10
11
7
\end{verbatim}

{\color{red}
\subsubsection{Sample Input 3}\label{sample-input-3}
\begin{verbatim}
2 3
walnut
walnut
1 walnut
1 walnut
2 walnut
\end{verbatim}


\subsubsection{Sample Output 3}\label{sample-output-3}
\begin{verbatim}
1
3
6
3
\end{verbatim}
}